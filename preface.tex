

\begin{center}
\large{\textbf{คำนำ}}
\end{center}
\addcontentsline{toc}{chapter}{คำนำ}

เอกสารประกอบการสอน รายวิชา 244211 กลศาสตร์ 1 (Mechanics I) จำนวนหน่วยกิต 3(3-0-6) หน่วยกิต เป็นรายวิชาที่เปิดสอนสำหรับนิสิต หลักสูตรวิทยาศาสตรบัณฑิต สาขาวิชาฟิสิกส์ และนิสิตหลักสูตรการศึกษาบัณฑิต สาขาวิชาการศึกษา และหลักสูตรวิทยาศาสตรบัณฑิต สาขาวิชาฟิสิกส์ ชั้นปี 2 มหาวิทยาลัยพะเยา โดยมี ข้าพเจ้า นายพงศพัศ แรงดี เป็นผู้สอน และผู้จัดการรายวิชา

ในเอกสารประกอบการสอนนี้ เป็นเนื้อหาที่ใช้ในการเรียนการสอนของวิชากลศาสตร์ 1 ซึ่งเนื้อหาต่างๆนั้น จะแยกเป็นบทๆ พร้อมทั้งมีแบบฝึกหัดให้ลองทำในท้ายหัวข้อย่อยแต่ละหัวข้อ โดยหัวข้อหลักๆ ประกอบไปด้วย เวกเตอร์ และการดำเนินการเวกเตอร์ขั้นสูง กฎการเคลื่อนที่ของนิวตัน กลศาสตร์นิวตันของอนุภาคเดี่ยวใน 1 มิติ 2 มิติ และ 3 มิติ การสั่น แรงศูนย์กลาง การเคลื่อนที่ในกรอบอ้างอิงไม่เฉื่อย การเคลื่อนที่ของระบบอนุภาค การเคลื่อนที่ของวัตถุแข็งเกร็ง กลศาสตร์ของไหล และจบเนื้อหาในเรื่อง กลศาสตร์แบบลากรองจ์และกลศาสตร์แบบแฮมิลตันเบื้องต้น และเนื้อหาทั้งหมดนี้ เป็นเนื้อหาที่ใช้ในการเรียนการสอนสำหรับ 1 ภาคการศึกษา

ผู้เขียนมีความคาดหวังว่า นิสิตที่อ่านเอกสารประกอบการสอนชุดนี้จะได้รับประโยชน์แก่ตนเอง โดยทำให้การเรียนวิชากลศาสตร์ 1 มีความเข้าใจมากยิ่งขึ้น และช่วยให้ทำคะแนนสอบได้ดีขึ้น อีกทั้งจะเป็นพื้นฐานสำหรับการเรียนรู้ในรายวิชาอื่นๆในระดับที่สูงขึ้นไปได้

หากท่านใดที่มีความสนใจจะศึกษาเพิ่มเติม นอกเหนือจากเนื้อหาในเอกสารประกอบการสอนเล่มนี้แล้ว ท่านสามารถค้นคว้าตามเอกสารในบรรณานุกรมได้ และหากมีข้อผิดพลาดประการใดผู้เขียนขอน้อมรับไว้แต่ผู้เดียว โดยหวังเป็นอย่างยิ่งว่าในฉบับพิมพ์ครั้งต่อไปจะพยายามลดข้อผิดพลาดให้เหลือน้อยที่สุด

%% Please "sign" your preface
\vspace{1cm}
\begin{flushleft}\noindent
มกราคม 2560 \hfill  พงศพัศ แรงดี
\end{flushleft}










