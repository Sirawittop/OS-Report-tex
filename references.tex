% การเขียนเอกสารอ้างอิง ให้ใช้รูปแบบ APA7 
% ลักษณะของการเขียนนี้ เป็นแบบ bibtex หากใช้รูปแบบอื่น
% ให้ปรับเป็นแบบ APA7 
%
% หนังสือทั่วไป ให้ใช้รูปแบบ โดยชื่อเรื่องต้องเป็นตัวเอียง ผู้แต่ง 1 - 20 คน ให้ใส่ชื่อทุกคน
%
% ผู้แต่ง. (ปีที่พิมพ์). \textit{ชื่อเรื่อง} (ครั้งที่พิมพ์ พิมพ์ครั้งที่ 2 เป็นต้นไป). สำนักพิมพ์.
%
% เช่น ภาสกร เนตรทิพย์วัลย์, พรพรรณ ภูสาหัส, และวิถี ธุระธรรม. (2565). \textit{การตรวจร่างกาย} (พิมพ์ครั้งที่ 2). พิมพ์ดีการพิมพ์.
% หรือ Kee, J. L., Marshall, S. M., \& Forrester, M. C. (2021). \textit{Clinical calculations} (9th ed.). Elsevier.
%
% ปล. ถ้าไม่ปรากฏปีที่พิมพ์ให้ใส่ (ม.ป.ป.) สำหรับภาษาไทย และ (n.d.) สำหรับภาษาอังกฤษ
%
% วารสารงานวิจัย ให้ใช้รูปแบบดังนี้
%
% ชื่อผู้เขียนบทความ. (ปีที่พิมพ์). ชื่อบทความ. \textit{ชื่อวารสาร}, \textit{เลขฉบับที่ หรือ Volume}(ฉบับที่ หรือ issue), หน้าแรก-หน้าสุดท้าย.
%
% เช่น บุศรา ชัยทัศน์. (2559). การดูแลผู้ป่วยโรงมะเร็งลำไส้ใหญ่. \textit{วารสารพยาบาลสภากาชาดไทย}, \textit{9}(1),19-33.
%หรือ Plows, J. F., Stanley, J. L., \& Vickers, M. H. (2018). The pathophysiology of gestational diabetes metllitus. \textit{International journal of molecular sciences}, \textit{19}(11), 3342.
% https:/doi.org/10.3390/ijms19113342 [arXiv:2210.07273 [astro-ph.HE]]
%
% ปล. ต้องใส่เลข doi ในรูปแบบลิงค์ https และอ้างอิง arXiv ใน [...] ด้วย ถ้ามี
%
% ดูเพิ่มเติมจาก https://tinyurl.com/4p6c5mf5

\begin{thebibliography}{99}
%\addcontentsline{toc}{chapter}{\numberline{}\textbf{บรรณานุกรม}}

\bibitem{ประวัติความเป็นมา:1}
bestinternet.co.th (2561). \textbf{Windows Server คืออะไร} เข้าถึงเมื่อ 16 กุมภาพันธ์. เข้าถึงได้จาก  https://www.bestinternet.co.th/single\_blog.php?id=94

\bibitem{ประวัติความเป็นมา:2}
addin.co.th (2565). \textbf{Windows Server คืออะไร มีกี่ประเภท กี่แบบ กี่เวอร์ชั่น และมีวิธีเลือกซื้อมาใช้งานอย่างไรบ้าง} เข้าถึงเมื่อ 16 กุมภาพันธ์. เข้าถึงได้จาก https://addin.co.th/blog/windows-server/

\bibitem{ประวัติความเป็นมา:Windows Server}
techtarget.com (2560). \textbf{Microsoft Windows Server OS (operating system)} เข้าถึงเมื่อ 16 กุมภาพันธ์. เข้าถึงได้จาก https://www.techtarget.com/searchwindowsserver/definition/Microsoft-Windows-Server-OS-operating-system

\bibitem{ประวัติความเป็นมา:2019}
blognone.com (2561). \textbf{Windows Server 2019 ออกตัวจริงแล้ว} เข้าถึงเมื่อ 16 กุมภาพันธ์. เข้าถึงได้จาก https://www.blognone.com/node/105700

\bibitem{2022:1}
learn.microsoft.com (2566). \textbf{What's new in Windows Server 2022} เข้าถึงเมื่อ 16 กุมภาพันธ์. เข้าถึงได้จาก
https://learn.microsoft.com/th-th/windows-server/get-started/whats-new-in-windows-server-2022

\bibitem{2022:2}
blognone.com (2564). \textbf{Windows Server 2022 ยกระดับความปลอดภัย, เน้นทำงานไฮบริด, อิมเมจขนาดเล็กลง} เข้าถึงเมื่อ 16 กุมภาพันธ์. เข้าถึงได้จาก
https://www.blognone.com/node/121489

\bibitem{2022:3}
learn.microsoft.com (2566). \textbf{What is Azure Edition for Windows Server?} เข้าถึงเมื่อ 16 กุมภาพันธ์. เข้าถึงได้จาก
https://learn.microsoft.com/en-us/windows-server/get-started/azure-edition 

\bibitem{job:1}
th.jobsdb.com (2566). \textbf{System Administrator ผู้ดูแลและจัดการระบบคอมพิวเตอร์ที่ทุกบริษัทควรต้องมี} เข้าถึงเมื่อ 16 กุมภาพันธ์. เข้าถึงได้จาก
https://th.jobsdb.com/th/career-advice/article/system-administrator

\clearpage
\bibitem{job:2}
th.jobsdb.com (2566). \textbf{วิศวกร DevOps เป็นผู้ที่เชื่อมขั้นตอนการพัฒนาซอฟต์แวร์ และนำขึ้นไปทดลองใช้ พร้อมสร้างระบบโครงสร้างพื้นฐานของซอฟต์แวร์ ให้ใช้งานได้อย่างราบรื่น} เข้าถึงเมื่อ 16 กุมภาพันธ์. เข้าถึงได้จาก
https://th.jobsdb.com/th/career-advice/role/devops-engineer


\bibitem{job:3}
th.jobsdb.com (2566). \textbf{เจาะลึก นักพัฒนาซอฟต์แวร์ (Software Developer) อาชีพในฝันของคนรุ่นใหม่} เข้าถึงเมื่อ 16 กุมภาพันธ์. เข้าถึงได้จาก
https://th.jobsdb.com/th/career-advice/article/software-developer

\bibitem{job:4}
bu.ac.th (2564). \textbf{เจาะลึกอาชีพผู้เชี่ยวชาญความปลอดภัยไซเบอร์ คณะเทคโนโลยีสารสนเทศและนวัตกรรม} เข้าถึงเมื่อ 16 กุมภาพันธ์. เข้าถึงได้จาก
https://www.bu.ac.th/th/featured-stories/636


\bibitem{job:5}
9experttraining.com (ม.ป.ป.). \textbf{รู้จักกับ Data Analyst และทักษะที่ต้องรู้} เข้าถึงเมื่อ 16 กุมภาพันธ์. เข้าถึงได้จาก
https://www.9experttraining.com/articles/data-analyst-\%E0\%B8\%84\%E0\%B8\%B7\%E0\%B8\%AD\%E0\%B9\%83\%E0\%B8\%84\%E0\%B8\%A3
\end{thebibliography}